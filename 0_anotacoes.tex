\documentclass[11pt]{article}%Template: https://www.overleaf.com/latex/templates/chapter-review-notes/npqqbrvfkwqh
\usepackage[utf8]{inputenc}
\usepackage[portuguese]{babel}
\usepackage{amsmath,amsthm,amsfonts,amssymb,amscd}
\usepackage{multirow,booktabs}
\usepackage[table]{xcolor}
\usepackage{fullpage}
\usepackage{lastpage}
\usepackage{enumitem}
\usepackage{fancyhdr}
\usepackage{mathrsfs}
\usepackage{wrapfig}
\usepackage{setspace}
\usepackage{calc}
\usepackage{multicol}
\usepackage{cancel}
\usepackage[retainorgcmds]{IEEEtrantools}
\usepackage[margin=3cm]{geometry}
\usepackage{amsmath}
\newlength{\tabcont}
\setlength{\parindent}{0.0in}
\setlength{\parskip}{0.05in}
\usepackage{empheq}
\usepackage{framed}
\usepackage[most]{tcolorbox}
\usepackage{xcolor}
\colorlet{shadecolor}{orange!15}
\parindent 0in
\parskip 12pt
\geometry{margin=1in, headsep=0.25in}
\theoremstyle{definition}
\newtheorem{defn}{Definition}
\newtheorem{reg}{Rule}
\newtheorem{exer}{Exercise}
\newtheorem{note}{Note}
\begin{document}
%\setcounter{section}{8}

%% INICIA LA DEFINICION DE COMANDOS %%}
\newcommand{\nota}[1]{% escribe una nota en rojo que aparece en el pdf
\textcolor{red}{
#1
}}
\newcommand{\titulo}{Anotações de Aula}
\newcommand{\griffiths}{Eletrodinâmica 3a Ed. J. Griffiths}

\title{\titulo}

\thispagestyle{empty}

\begin{center}
{\LARGE \bf \titulo}\\
{\large Eletrodinâmica 3a Ed. J. Griffiths}\\
Edital 045-2020 EP-USP
\end{center}

\section[Teorema de Poynting]{Teorema de Poynting e densidades de energia armazenada em campos elétricos e magnéticos.\footnote{Cap. 8 de \griffiths}
}

Como foi visto na seção de eletrostática, a a carga em um volume $V$ é
\begin{equation}
  \label{carga}
Q(t) = \int_{V} \rho({\vec r},t)d\tau ,
\end{equation}
onde $d\tau$ é um elemento de volume infinitesimal da distribuição de cargas em $V$. Isto é, integramos a densidade de carga $\rho$ no volume $V$ para obter a carga total $Q$.

Sempre que definimos um volume $V$, temos um envoltório que define esse volume. Assim como o volume do nosso planeta poderi


\subsection{Acceleration Without Rotation}
Consider an inertial reference frame (i.e not accelerating) which will be denoted S$_0$, and a accelerating reference frame, \textit{S} that has an acceleration of \textit{A}.
\begin{note}
\textbf{Capital Letters refer to the accelerating reference frame \textit{S} while lowercase letters refer to the inertial reference frame S$_0$}
\end{note}
Picture a moving reference frame, \textit{S}, moving relative to S$_0$. Imagine in the the moving reference frame that a ball with mass, \textit{m} is being thrown. 
In order to consider the motion of the ball, the motion must be first considered in the inertial reference frame. 
\begin{equation}
F = m\ddot{r_0}
\end{equation}
Where r$_0$ is the ball's position relative to S$_0$. 

Now, by considering the motion of the ball in the accelerating frame, the ball position relative to \textit{S} is \textit{R}. (It's velocity is $\dot{R}$. 
Thus, relating \textit{R} to $r_0$, we have: 
\begin{equation}
\dot{r_0} = \dot{R} + V
\end{equation}
Newton's second law for the inertial reference frame by differentiate and multiplying by mass is:
\begin{equation}
F_{\text{inertial}} = -mA = -m\ddot{R}
\end{equation}

\input{0_Apagar.tex}% texto do documento original
\end{document}