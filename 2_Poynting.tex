\section[Teorema de Poynting]{Teorema de Poynting e densidades de energia armazenada em campos elétricos e magnéticos.\footnote{Cap. 8 de \griffiths}
}

Como foi visto na seção de eletrostática, a a carga em um volume $V$ é
\begin{equation}
  \label{carga}
Q(t) = \int_{V} \rho({\vec r},t)d\tau ,
\end{equation}
onde $d\tau$ é um elemento de volume infinitesimal da distribuição de cargas em $V$. Isto é, integramos a densidade de carga $\rho$ no volume $V$ para obter a carga total $Q$.

Sempre que definimos um volume $V$, temos um envoltório que define esse volume. Assim como o volume do nosso planeta poderi


\subsection{Acceleration Without Rotation}
Consider an inertial reference frame (i.e not accelerating) which will be denoted S$_0$, and a accelerating reference frame, \textit{S} that has an acceleration of \textit{A}.
\begin{note}
\textbf{Capital Letters refer to the accelerating reference frame \textit{S} while lowercase letters refer to the inertial reference frame S$_0$}
\end{note}
Picture a moving reference frame, \textit{S}, moving relative to S$_0$. Imagine in the the moving reference frame that a ball with mass, \textit{m} is being thrown. 
In order to consider the motion of the ball, the motion must be first considered in the inertial reference frame. 
\begin{equation}
F = m\ddot{r_0}
\end{equation}
Where r$_0$ is the ball's position relative to S$_0$. 

Now, by considering the motion of the ball in the accelerating frame, the ball position relative to \textit{S} is \textit{R}. (It's velocity is $\dot{R}$. 
Thus, relating \textit{R} to $r_0$, we have: 
\begin{equation}
\dot{r_0} = \dot{R} + V
\end{equation}
Newton's second law for the inertial reference frame by differentiate and multiplying by mass is:
\begin{equation}
F_{\text{inertial}} = -mA = -m\ddot{R}
\end{equation}